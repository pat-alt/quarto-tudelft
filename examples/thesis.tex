% Options for packages loaded elsewhere
\PassOptionsToPackage{unicode}{hyperref}
\PassOptionsToPackage{hyphens}{url}
\PassOptionsToPackage{dvipsnames,svgnames,x11names}{xcolor}
%
\documentclass{tudelft}

\usepackage{amsmath,amssymb}
\usepackage{iftex}
\ifPDFTeX
  \usepackage[T1]{fontenc}
  \usepackage[utf8]{inputenc}
  \usepackage{textcomp} % provide euro and other symbols
\else % if luatex or xetex
  \usepackage{unicode-math}
  \defaultfontfeatures{Scale=MatchLowercase}
  \defaultfontfeatures[\rmfamily]{Ligatures=TeX,Scale=1}
\fi
\usepackage{lmodern}
\ifPDFTeX\else  
    % xetex/luatex font selection
\fi
% Use upquote if available, for straight quotes in verbatim environments
\IfFileExists{upquote.sty}{\usepackage{upquote}}{}
\IfFileExists{microtype.sty}{% use microtype if available
  \usepackage[]{microtype}
  \UseMicrotypeSet[protrusion]{basicmath} % disable protrusion for tt fonts
}{}
\makeatletter
\@ifundefined{KOMAClassName}{% if non-KOMA class
  \IfFileExists{parskip.sty}{%
    \usepackage{parskip}
  }{% else
    \setlength{\parindent}{0pt}
    \setlength{\parskip}{6pt plus 2pt minus 1pt}}
}{% if KOMA class
  \KOMAoptions{parskip=half}}
\makeatother
\usepackage{xcolor}
\setlength{\emergencystretch}{3em} % prevent overfull lines
\setcounter{secnumdepth}{-\maxdimen} % remove section numbering
% Make \paragraph and \subparagraph free-standing
\makeatletter
\ifx\paragraph\undefined\else
  \let\oldparagraph\paragraph
  \renewcommand{\paragraph}{
    \@ifstar
      \xxxParagraphStar
      \xxxParagraphNoStar
  }
  \newcommand{\xxxParagraphStar}[1]{\oldparagraph*{#1}\mbox{}}
  \newcommand{\xxxParagraphNoStar}[1]{\oldparagraph{#1}\mbox{}}
\fi
\ifx\subparagraph\undefined\else
  \let\oldsubparagraph\subparagraph
  \renewcommand{\subparagraph}{
    \@ifstar
      \xxxSubParagraphStar
      \xxxSubParagraphNoStar
  }
  \newcommand{\xxxSubParagraphStar}[1]{\oldsubparagraph*{#1}\mbox{}}
  \newcommand{\xxxSubParagraphNoStar}[1]{\oldsubparagraph{#1}\mbox{}}
\fi
\makeatother


\providecommand{\tightlist}{%
  \setlength{\itemsep}{0pt}\setlength{\parskip}{0pt}}\usepackage{longtable,booktabs,array}
\usepackage{calc} % for calculating minipage widths
% Correct order of tables after \paragraph or \subparagraph
\usepackage{etoolbox}
\makeatletter
\patchcmd\longtable{\par}{\if@noskipsec\mbox{}\fi\par}{}{}
\makeatother
% Allow footnotes in longtable head/foot
\IfFileExists{footnotehyper.sty}{\usepackage{footnotehyper}}{\usepackage{footnote}}
\makesavenoteenv{longtable}
\usepackage{graphicx}
\makeatletter
\newsavebox\pandoc@box
\newcommand*\pandocbounded[1]{% scales image to fit in text height/width
  \sbox\pandoc@box{#1}%
  \Gscale@div\@tempa{\textheight}{\dimexpr\ht\pandoc@box+\dp\pandoc@box\relax}%
  \Gscale@div\@tempb{\linewidth}{\wd\pandoc@box}%
  \ifdim\@tempb\p@<\@tempa\p@\let\@tempa\@tempb\fi% select the smaller of both
  \ifdim\@tempa\p@<\p@\scalebox{\@tempa}{\usebox\pandoc@box}%
  \else\usebox{\pandoc@box}%
  \fi%
}
% Set default figure placement to htbp
\def\fps@figure{htbp}
\makeatother

\usepackage{xstring}
\makeatletter
\@ifpackageloaded{caption}{}{\usepackage{caption}}
\AtBeginDocument{%
\ifdefined\contentsname
  \renewcommand*\contentsname{Table of contents}
\else
  \newcommand\contentsname{Table of contents}
\fi
\ifdefined\listfigurename
  \renewcommand*\listfigurename{List of Figures}
\else
  \newcommand\listfigurename{List of Figures}
\fi
\ifdefined\listtablename
  \renewcommand*\listtablename{List of Tables}
\else
  \newcommand\listtablename{List of Tables}
\fi
\ifdefined\figurename
  \renewcommand*\figurename{Figure}
\else
  \newcommand\figurename{Figure}
\fi
\ifdefined\tablename
  \renewcommand*\tablename{Table}
\else
  \newcommand\tablename{Table}
\fi
}
\@ifpackageloaded{float}{}{\usepackage{float}}
\floatstyle{ruled}
\@ifundefined{c@chapter}{\newfloat{codelisting}{h}{lop}}{\newfloat{codelisting}{h}{lop}[chapter]}
\floatname{codelisting}{Listing}
\newcommand*\listoflistings{\listof{codelisting}{List of Listings}}
\makeatother
\makeatletter
\makeatother
\makeatletter
\@ifpackageloaded{caption}{}{\usepackage{caption}}
\@ifpackageloaded{subcaption}{}{\usepackage{subcaption}}
\makeatother
\makeatletter
\@ifpackageloaded{algorithm}{}{\usepackage{algorithm}}
\makeatother
\makeatletter
\@ifpackageloaded{algpseudocode}{}{\usepackage{algpseudocode}}
\makeatother
\makeatletter
\@ifpackageloaded{caption}{}{\usepackage{caption}}
\makeatother
\makeatletter
\@ifpackageloaded{tcolorbox}{}{\usepackage[many]{tcolorbox}}
\makeatother
%%%% ---foldboxy preamble ----- %%%%%

\definecolor{fbx-default-color1}{HTML}{c7c7d0}
\definecolor{fbx-default-color2}{HTML}{a3a3aa}

\definecolor{fbox-color1}{HTML}{c7c7d0}
\definecolor{fbox-color2}{HTML}{a3a3aa}

% arguments: #1 typelabelnummer: #2 titel: #3
\newenvironment{fbx}[3]{\begin{tcolorbox}[enhanced, breakable,%
attach boxed title to top*={xshift=1.4pt},
boxed title style={boxrule=0.0mm, fuzzy shadow={1pt}{-1pt}{0mm}{0.1mm}{gray}, arc=.3em, rounded corners=east, sharp corners=west}, colframe=#1-color2, colbacktitle=#1-color1, colback = white, coltitle=black,  titlerule=0mm, toprule=0pt, bottomrule=.7pt, leftrule=.3em, rightrule=0pt, outer arc=.3em,  arc=0pt,	 sharp corners = east, left=.5em, bottomtitle=1mm, toptitle=1mm,title=\textbf{#2}\hspace{0.5em}{#3}]}
{\end{tcolorbox}}

% boxed environment with right border
\newenvironment{fbxSimple}[3]{\begin{tcolorbox}[enhanced, breakable,%
attach boxed title to top*={xshift=1.4pt},
boxed title style={boxrule=0.0mm, fuzzy shadow={1pt}{-1pt}{0mm}{0.1mm}{gray}, arc=.3em, rounded corners=east, sharp corners=west}, colframe=#1-color2, colbacktitle=#1-color1, colback = white, coltitle=black,  titlerule=0mm, toprule=0pt, bottomrule=.7pt, leftrule=.3em, rightrule=.7pt, outer arc=.3em,  	left=.5em, right=.5em, bottomtitle=1mm, toptitle=1mm,title=\textbf{#2}\hspace{0.5em}{#3}]}
{\end{tcolorbox}}

%%%% --- end foldboxy preamble ----- %%%%%
%%==== colors from yaml ===%
\definecolor{RQ-color1}{HTML}{dff4fa}
\definecolor{RQ-color2}{HTML}{00a6d6}
\definecolor{TRQ-color1}{HTML}{80d2eb}
\definecolor{TRQ-color2}{HTML}{00a6d6}
\definecolor{Rec-color1}{HTML}{fff6e3}
\definecolor{Rec-color2}{HTML}{ffb81c}
%=============%

\usepackage{bookmark}

\IfFileExists{xurl.sty}{\usepackage{xurl}}{} % add URL line breaks if available
\urlstyle{same} % disable monospaced font for URLs
\hypersetup{
  pdftitle={TU Delft PhD Thesis Template},
  pdfauthor={Patrick Altmeyer},
  pdfkeywords={Quarto, Extension, Template},
  colorlinks=true,
  linkcolor={blue},
  filecolor={Maroon},
  citecolor={Blue},
  urlcolor={Blue},
  pdfcreator={LaTeX via pandoc}}


\title{TU Delft PhD Thesis Template}
\author{Patrick Altmeyer}
\date{}

\begin{document}
% Adapted with the help of Claude Sonnet 3.7
% !TeX spellcheck = en_GB

% Apply frontmatter styling (Roman numerals for the page numbers of the title pages and toc)
\frontmatter

\begin{titlepage}

    \begin{center}

        %% Extra whitespace at the top.
        \vspace*{2\bigskipamount}

        %% Print the title.
        {\makeatletter
            \titlestyle\bfseries\LARGE\@title
            \makeatother}

        %% Print the optional subtitle.
        {\makeatletter
            \ifx\@subtitle\undefined\else
                \bigskip
                \titlefont\titleshape\Large\@subtitle
            \fi
            \makeatother}

    \end{center}

    \cleardoublepage
    \thispagestyle{empty}

    \begin{center}

        %% The following lines repeat the previous page exactly.

        \vspace*{2\bigskipamount}

        %% Print the title.
        {\makeatletter
            \titlestyle\bfseries\LARGE\@title
            \makeatother}

        %% Print the optional subtitle.
        {\makeatletter
            \ifx\@subtitle\undefined\else
                \bigskip
                \titlefont\titleshape\Large\@subtitle
            \fi
            \makeatother}

        %% Uncomment the following lines to insert a vertically centered picture into
        %% the title page.
        %\vfill
        %\includegraphics{title}
        \vfill

        %% Apart from the names and dates, the following text is dictated by the
        %% promotieregelement.

        {\Large\titlefont\bfseries Dissertation}

        \bigskip
        \bigskip

        for the purpose of obtaining the degree of doctor

        at Delft University of Technology

        by the authority of the Rector Magnificus, Prof.~dr.~ir.~T.H.J.J.~van der Hagen,

        Chair of the Board for Doctorates

        to be defended publicly on

        Tuesday 20, June 2025 at 10:00 o'clock

        \bigskip
        \bigskip

        by

        \bigskip
        \bigskip

        %% Print the full name of the author.
                    \newcommand{\formatname}[2]{#1 \MakeUppercase{#2}} % First Last -> First LAST
            \StrBefore{Patrick Altmeyer}{ }[\FirstName]
            \StrBehind{Patrick Altmeyer}{ }[\LastName]
            {\Large\titlefont\bfseries \formatname{\FirstName}{\LastName}}
        
        \bigskip
        \bigskip

                MSc Research, University of Research, Country
        
        born in Amsterdam, The Netherlands

        %% Extra whitespace at the bottom.
        \vspace*{2\bigskipamount}

    \end{center}

    \clearpage
    \thispagestyle{empty}

    %% The following line is dictated by the promotieregelement.
    \noindent This dissertation has been approved by the promotors.

    \bigskip
    \noindent Composition of the doctoral committee:
    %% List the committee members, starting with the Rector Magnificus and the
    %% promotor(s) and ending with the reserve members.
    \begin{tabbing}
        \hspace{\tabcolsep}\=\hspace{0.33\textwidth}\=\hspace{0.66\textwidth}                   \\[-3\medskipamount]
        \> Rector Magnificus,          \> chairperson\\

        \>Prof.~dr. A. Kleiner       \> Delft University of
Technology, promotor                         \\
        \>Dr.~A.A. Aaronson       \> Delft University of
Technology, copromotor                         \\
        \>Prof.~dr. ir. A.B.C.D. van de Lange-Achternaam\\
        \>       \> Delft University of
Technology                         \\

        \>\textit{Independent members:}                                                        \\[\smallskipamount]
        \>Prof.~dr. A. Jansen       \> Delft University of
Technology                         \\
        \>Prof.~dr. ir. A.B.C.D. van de Andere Lange-Achternaam\\
        \>       \> Delft University of
Technology                         \\
        \>Prof.~dr. N. Nescio       \> Politecnico di Milano,
Italy                         \\
        \>Prof.~dr. ir. J. Doe       \> Delft University of
Technology, reserve member                         \\
        \>\textit{Other members:}                                                               \\[\smallskipamount]
        \>Prof.~dr. ir. J. de Wit, \> Delft University of
Technology                         \\
        \>Dr.~ir. Q. de Zwart, \> Delft University of
Technology                         \\
    \end{tabbing}

    %% Include the following disclaimer for committee members who have contributed
    %% to this dissertation. Its formulation is again dictated by the
    %% promotieregelement.

    \medskip
    \noindent This is an optional message to acknowledge partner
institutes.

    %% Here you can include the logos of any institute that contributed financially
    %% to this dissertation.
    \vfill
        \begin{center}
                                \includegraphics[height=0.75in]{www/tudelft.png}\hspace{1.5cm}
                        \includegraphics[height=0.75in]{www/delft-final.png}\hspace{1cm}
                        \end{center}
    \vfill

    %% Here you can include the logos of any institute that contributed financially
    %% to this dissertation.
    \vfill
    \begin{center}
    \end{center}
    \vfill

    \noindent
    \begin{tabular}{@{}p{0.2\textwidth}@{}p{0.8\textwidth}@{}}
        \textit{Keywords:}    & Quarto, Extension, Template \\[\medskipamount]
        \textit{Printed by:}   & Johannes Gutenberg \\[\medskipamount]
        \textit{Cover by:} & Beautiful cover art that captures the entire content of this thesis in a single illustration.
    \end{tabular}

    \vspace{4\bigskipamount}



    \medskip
    \noindent The author set this thesis using Quarto: \url{https://github.com/quarto-tudelft}.

    \medskip
    \noindent ISBN 000-00-0000-000-0

    \medskip
    \noindent An electronic copy of this dissertation is available at\\
    \url{https://repository.tudelft.nl/}.

\end{titlepage}

%% The (optional) dedication can be used to thank someone or display a
%% significant quotation.
\dedication{
      \epigraph{They not like us.}{Kendrik Lamar}
  }

{
  \cleardoublepage%
  \phantomsection%
}
\floatname{algorithm}{Algorithm}

\renewcommand*\contentsname{Table of contents}
{
\hypersetup{linkcolor=}
\setcounter{tocdepth}{3}
\tableofcontents
}

\section*{Preface}\label{preface}
\addcontentsline{toc}{section}{Preface}

\section*{Summary}\label{summary}
\addcontentsline{toc}{section}{Summary}

\section*{Samenvatting}\label{samenvatting}
\addcontentsline{toc}{section}{Samenvatting}

\pagebreak
\mainmatter

\section{Introduction}\label{introduction}

This is a Quarto extension for generating your TU Delft PhD Thesis in
Quarto.

\section{Cool Quarto Things}\label{cool-quarto-things}

\subsection{Custom Callouts}\label{custom-callouts}

The template adds custom numbered blocks using the embedded
\href{https://github.com/ute/custom-numbered-blocks}{ute/custom-numbered-blocks}
extension, which supports HTML and PDF with non-standard
cross-references (i.e.~\texttt{\textbackslash{}ref\{\}} syntax). We
provide two custom numbered blocks:

We provide two custom numbered blocks:

\begin{enumerate}
\def\labelenumi{\arabic{enumi}.}
\tightlist
\item
  \textbf{TRQ} for Thesis Research Questions
\item
  \textbf{RQ} for Research Questions
\item
  \textbf{Rec} for Recommendations
\end{enumerate}

TRQ \hyperref[trq:what]{2.1} is an example of a Thesis Research
Question; RQ \hyperref[rq:what]{2.1} is an example of a Research
Questions; Rec. \hyperref[rec:what]{2.1} is an example of a
recommendation.

\phantomsection\label{trq:what}
\begin{fbx}{TRQ}{Thesis Research Question 2.1: }{What is a TRQ?}
\phantomsection\label{trq:what}
This is a Thesis Research Question, that you might want to use in the
introduction of your thesis.

\end{fbx}

\phantomsection\label{rq:what}
\begin{fbx}{RQ}{Research Question 2.1: }{What is a RQ?}
\phantomsection\label{rq:what}
This is a Research Question, that you might want to use in any chapter
of your thesis.

\end{fbx}

\phantomsection\label{rec:what}
\begin{fbx}{Rec}{Recommendation 2.1: }{What is a Rec?}
\phantomsection\label{rec:what}
This is a Recommendation that you might want to use in the conclusion of
your thesis.

\end{fbx}




\end{document}
