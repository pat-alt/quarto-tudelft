% Options for packages loaded elsewhere
% Options for packages loaded elsewhere
\PassOptionsToPackage{unicode}{hyperref}
\PassOptionsToPackage{hyphens}{url}
\PassOptionsToPackage{dvipsnames,svgnames,x11names}{xcolor}
%
\documentclass{tudelft}
\usepackage{xcolor}
\usepackage{amsmath,amssymb}
\setcounter{secnumdepth}{-\maxdimen} % remove section numbering
\usepackage{iftex}
\ifPDFTeX
  \usepackage[T1]{fontenc}
  \usepackage[utf8]{inputenc}
  \usepackage{textcomp} % provide euro and other symbols
\else % if luatex or xetex
  \usepackage{unicode-math} % this also loads fontspec
  \defaultfontfeatures{Scale=MatchLowercase}
  \defaultfontfeatures[\rmfamily]{Ligatures=TeX,Scale=1}
\fi
\usepackage{lmodern}
\ifPDFTeX\else
  % xetex/luatex font selection
\fi
% Use upquote if available, for straight quotes in verbatim environments
\IfFileExists{upquote.sty}{\usepackage{upquote}}{}
\IfFileExists{microtype.sty}{% use microtype if available
  \usepackage[]{microtype}
  \UseMicrotypeSet[protrusion]{basicmath} % disable protrusion for tt fonts
}{}
\makeatletter
\@ifundefined{KOMAClassName}{% if non-KOMA class
  \IfFileExists{parskip.sty}{%
    \usepackage{parskip}
  }{% else
    \setlength{\parindent}{0pt}
    \setlength{\parskip}{6pt plus 2pt minus 1pt}}
}{% if KOMA class
  \KOMAoptions{parskip=half}}
\makeatother
% Make \paragraph and \subparagraph free-standing
\makeatletter
\ifx\paragraph\undefined\else
  \let\oldparagraph\paragraph
  \renewcommand{\paragraph}{
    \@ifstar
      \xxxParagraphStar
      \xxxParagraphNoStar
  }
  \newcommand{\xxxParagraphStar}[1]{\oldparagraph*{#1}\mbox{}}
  \newcommand{\xxxParagraphNoStar}[1]{\oldparagraph{#1}\mbox{}}
\fi
\ifx\subparagraph\undefined\else
  \let\oldsubparagraph\subparagraph
  \renewcommand{\subparagraph}{
    \@ifstar
      \xxxSubParagraphStar
      \xxxSubParagraphNoStar
  }
  \newcommand{\xxxSubParagraphStar}[1]{\oldsubparagraph*{#1}\mbox{}}
  \newcommand{\xxxSubParagraphNoStar}[1]{\oldsubparagraph{#1}\mbox{}}
\fi
\makeatother


\usepackage{longtable,booktabs,array}
\usepackage{calc} % for calculating minipage widths
% Correct order of tables after \paragraph or \subparagraph
\usepackage{etoolbox}
\makeatletter
\patchcmd\longtable{\par}{\if@noskipsec\mbox{}\fi\par}{}{}
\makeatother
% Allow footnotes in longtable head/foot
\IfFileExists{footnotehyper.sty}{\usepackage{footnotehyper}}{\usepackage{footnote}}
\makesavenoteenv{longtable}
\usepackage{graphicx}
\makeatletter
\newsavebox\pandoc@box
\newcommand*\pandocbounded[1]{% scales image to fit in text height/width
  \sbox\pandoc@box{#1}%
  \Gscale@div\@tempa{\textheight}{\dimexpr\ht\pandoc@box+\dp\pandoc@box\relax}%
  \Gscale@div\@tempb{\linewidth}{\wd\pandoc@box}%
  \ifdim\@tempb\p@<\@tempa\p@\let\@tempa\@tempb\fi% select the smaller of both
  \ifdim\@tempa\p@<\p@\scalebox{\@tempa}{\usebox\pandoc@box}%
  \else\usebox{\pandoc@box}%
  \fi%
}
% Set default figure placement to htbp
\def\fps@figure{htbp}
\makeatother





\setlength{\emergencystretch}{3em} % prevent overfull lines

\providecommand{\tightlist}{%
  \setlength{\itemsep}{0pt}\setlength{\parskip}{0pt}}



 


\makeatletter
\@ifpackageloaded{float}{}{\usepackage{float}}
\floatstyle{plain}
\@ifundefined{c@chapter}{\newfloat{trqthm}{h}{lotrq}}{\newfloat{trqthm}{h}{lotrq}[chapter]}
\floatname{trqthm}{TRQ}
\newcommand*\listoftrqthms{\listof{trqthm}{List of TRQs}}
\makeatother
\makeatletter
\@ifpackageloaded{float}{}{\usepackage{float}}
\floatstyle{plain}
\@ifundefined{c@chapter}{\newfloat{rq}{h}{lorq}}{\newfloat{rq}{h}{lorq}[chapter]}
\floatname{rq}{RQ}
\newcommand*\listofrqs{\listof{rq}{List of Research Questions}}
\makeatother
\makeatletter
\@ifpackageloaded{tcolorbox}{}{\usepackage[skins,breakable]{tcolorbox}}
\@ifpackageloaded{fontawesome5}{}{\usepackage{fontawesome5}}
\definecolor{quarto-callout-color}{HTML}{909090}
\definecolor{quarto-callout-note-color}{HTML}{0758E5}
\definecolor{quarto-callout-important-color}{HTML}{CC1914}
\definecolor{quarto-callout-warning-color}{HTML}{EB9113}
\definecolor{quarto-callout-tip-color}{HTML}{00A047}
\definecolor{quarto-callout-caution-color}{HTML}{FC5300}
\definecolor{quarto-callout-color-frame}{HTML}{acacac}
\definecolor{quarto-callout-note-color-frame}{HTML}{4582ec}
\definecolor{quarto-callout-important-color-frame}{HTML}{d9534f}
\definecolor{quarto-callout-warning-color-frame}{HTML}{f0ad4e}
\definecolor{quarto-callout-tip-color-frame}{HTML}{02b875}
\definecolor{quarto-callout-caution-color-frame}{HTML}{fd7e14}
\makeatother
\makeatletter
\@ifpackageloaded{caption}{}{\usepackage{caption}}
\AtBeginDocument{%
\ifdefined\contentsname
  \renewcommand*\contentsname{Table of contents}
\else
  \newcommand\contentsname{Table of contents}
\fi
\ifdefined\listfigurename
  \renewcommand*\listfigurename{List of Figures}
\else
  \newcommand\listfigurename{List of Figures}
\fi
\ifdefined\listtablename
  \renewcommand*\listtablename{List of Tables}
\else
  \newcommand\listtablename{List of Tables}
\fi
\ifdefined\figurename
  \renewcommand*\figurename{Figure}
\else
  \newcommand\figurename{Figure}
\fi
\ifdefined\tablename
  \renewcommand*\tablename{Table}
\else
  \newcommand\tablename{Table}
\fi
}
\@ifpackageloaded{float}{}{\usepackage{float}}
\floatstyle{ruled}
\@ifundefined{c@chapter}{\newfloat{codelisting}{h}{lop}}{\newfloat{codelisting}{h}{lop}[chapter]}
\floatname{codelisting}{Listing}
\newcommand*\listoflistings{\listof{codelisting}{List of Listings}}
\makeatother
\makeatletter
\makeatother
\makeatletter
\@ifpackageloaded{caption}{}{\usepackage{caption}}
\@ifpackageloaded{subcaption}{}{\usepackage{subcaption}}
\makeatother
\makeatletter
\@ifpackageloaded{algorithm}{}{\usepackage{algorithm}}
\makeatother
\makeatletter
\@ifpackageloaded{algpseudocode}{}{\usepackage{algpseudocode}}
\makeatother
\makeatletter
\@ifpackageloaded{caption}{}{\usepackage{caption}}
\makeatother
\makeatletter
\@ifpackageloaded{tcolorbox}{}{\usepackage[many]{tcolorbox}}
\makeatother
%%%% ---foldboxy preamble ----- %%%%%

\definecolor{fbx-default-color1}{HTML}{c7c7d0}
\definecolor{fbx-default-color2}{HTML}{a3a3aa}

\definecolor{fbox-color1}{HTML}{c7c7d0}
\definecolor{fbox-color2}{HTML}{a3a3aa}

% arguments: #1 typelabelnummer: #2 titel: #3
\newenvironment{fbx}[3]{\begin{tcolorbox}[enhanced, breakable,%
attach boxed title to top*={xshift=1.4pt},
boxed title style={boxrule=0.0mm, fuzzy shadow={1pt}{-1pt}{0mm}{0.1mm}{gray}, arc=.3em, rounded corners=east, sharp corners=west}, colframe=#1-color2, colbacktitle=#1-color1, colback = white, coltitle=black,  titlerule=0mm, toprule=0pt, bottomrule=.7pt, leftrule=.3em, rightrule=0pt, outer arc=.3em,  arc=0pt,	 sharp corners = east, left=.5em, bottomtitle=1mm, toptitle=1mm,title=\textbf{#2}\hspace{0.5em}{#3}]}
{\end{tcolorbox}}

% boxed environment with right border
\newenvironment{fbxSimple}[3]{\begin{tcolorbox}[enhanced, breakable,%
attach boxed title to top*={xshift=1.4pt},
boxed title style={boxrule=0.0mm, fuzzy shadow={1pt}{-1pt}{0mm}{0.1mm}{gray}, arc=.3em, rounded corners=east, sharp corners=west}, colframe=#1-color2, colbacktitle=#1-color1, colback = white, coltitle=black,  titlerule=0mm, toprule=0pt, bottomrule=.7pt, leftrule=.3em, rightrule=.7pt, outer arc=.3em,  	left=.5em, right=.5em, bottomtitle=1mm, toptitle=1mm,title=\textbf{#2}\hspace{0.5em}{#3}]}
{\end{tcolorbox}}

%%%% --- end foldboxy preamble ----- %%%%%
%%==== colors from yaml ===%
\definecolor{RQ-color1}{HTML}{ebfaeb}
\definecolor{RQ-color2}{HTML}{2eb82e}
\definecolor{TRQ-color1}{HTML}{adebad}
\definecolor{TRQ-color2}{HTML}{2eb82e}
%=============%
\newcounter{quartocalloutnteno}
\newcommand{\quartocalloutnte}[1]{\refstepcounter{quartocalloutnteno}\label{#1}}
\newcounter{quartocallouttrqno}
\newcommand{\quartocallouttrq}[1]{\refstepcounter{quartocallouttrqno}\label{#1}}
\newcounter{quartocalloutrqno}
\newcommand{\quartocalloutrq}[1]{\refstepcounter{quartocalloutrqno}\label{#1}}
\usepackage{bookmark}
\IfFileExists{xurl.sty}{\usepackage{xurl}}{} % add URL line breaks if available
\urlstyle{same}
\hypersetup{
  pdftitle={TU Delft PhD Thesis Template},
  pdfauthor={Patrick Altmeyer},
  pdfkeywords={template, demo},
  colorlinks=true,
  linkcolor={blue},
  filecolor={Maroon},
  citecolor={Blue},
  urlcolor={Blue},
  pdfcreator={LaTeX via pandoc}}


\title{TU Delft PhD Thesis Template}
\author{Patrick Altmeyer}
\date{}
\begin{document}
\maketitle
\begin{abstract}
This document is a template demonstrating the \texttt{thesis-tudelft}
format.
\end{abstract}

\floatname{algorithm}{Algorithm}


\section{Introduction}\label{introduction}

This is a Quarto extension for generating your TU Delft PhD Thesis in
Quarto.

\section{Cool Quarto Things}\label{cool-quarto-things}

\subsection{Custom Callouts}\label{custom-callouts}

The template adds custom numbered blocks and custom cross-references
that can be combined with standard callout blocks like
Note~\ref{nte-custom}.

\subsubsection{Custom Numbered Blocks}\label{custom-numbered-blocks}

The recommended approach is to rely on the embedded
\href{https://github.com/ute/custom-numbered-blocks}{ute/custom-numbered-blocks}
extension, which supports HTML and PDF with non-standard
cross-references (i.e.~\texttt{\textbackslash{}ref\{\}} syntax). We
provide two custom numbered blocks:

We provide two custom numbered blocks:

\begin{enumerate}
\def\labelenumi{\arabic{enumi}.}
\tightlist
\item
  \textbf{TRQ} for Thesis Research Questions
\item
  \textbf{RQ} for Research Questions
\end{enumerate}

TRQ \hyperref[trq:what]{2.1} is an example of the former; RQ
\hyperref[rq:what]{2.2} is an example of the latter.

\phantomsection\label{trq:what}
\begin{fbxSimple}{TRQ}{❔ TRQ 2.1: }{What is a TRQ?}
\phantomsection\label{trq:what}
This is a Thesis Research Question, that you might want to use in the
introduction of your thesis.

\end{fbxSimple}

\phantomsection\label{rq:what}
\begin{fbxSimple}{RQ}{RQ 2.2: }{What is a RQ?}
\phantomsection\label{rq:what}
This is a Research Question, that you might want to use in any chapter
of your thesis.

\end{fbxSimple}

\subsubsection{Callouts with Custom
Cross-References}\label{callouts-with-custom-cross-references}

Alternatively, you can rely on standard callout blocks combined with
custom cross-references. This approach has certain limitations
(Note~\ref{nte-custom}), but may be preferable if you like the default
callouts.

\begin{tcolorbox}[enhanced jigsaw, colframe=quarto-callout-note-color-frame, opacityback=0, leftrule=.75mm, left=2mm, bottomrule=.15mm, toprule=.15mm, colback=white, rightrule=.15mm, arc=.35mm, breakable]
\begin{minipage}[t]{5.5mm}
\textcolor{quarto-callout-note-color}{\faInfo}
\end{minipage}%
\begin{minipage}[t]{\textwidth - 5.5mm}

\quartocalloutnte{nte-custom} 

\vspace{-3mm}\textbf{Note \ref*{nte-custom}: Current Limitations}\vspace{3mm}

This is a makeshift solution surprisingly works, but it has certain
limitations. Firstly, it relies on setting the default behaviour of
callout blocks to \texttt{callout-appearance:\ simple} (to override
this, just use the YAML header of you document or the
\texttt{\_quarto.yml} config file). We rely on the default style of
callout blocks, e.g.~the \texttt{.callout-tip} for questions, where the
light bulb icon sort of makes sense. Secondly, it works well for PDF, in
that the title in the callout block is changed to TRQ/RQ, but not for
HTML. We use
\href{https://quarto.thecoatlessprofessor.com/custom-callout/qcustom-callout-faq.html}{coatless-quarto/custom-callouts}
to add a custom icon in HTML, but unfortunately we have not been able to
affect the title.

\end{minipage}%
\end{tcolorbox}

TRQ~\ref{trq-what} is a Thesis Research Question.

\begin{tcolorbox}[enhanced jigsaw, colframe=quarto-callout-tip-color-frame, opacityback=0, leftrule=.75mm, left=2mm, bottomrule=.15mm, toprule=.15mm, colback=white, rightrule=.15mm, arc=.35mm, breakable]
\begin{minipage}[t]{5.5mm}
\textcolor{quarto-callout-tip-color}{\faLightbulb}
\end{minipage}%
\begin{minipage}[t]{\textwidth - 5.5mm}

\quartocallouttrq{trq-what} 

\vspace{-3mm}\textbf{TRQ \ref*{trq-what}: What are Thesis Research Questions?}\vspace{3mm}

This is a Thesis Research Question, that you might want to use in the
introduction of your thesis. It can be cross-referenced using the custom
\texttt{trq} key.

\end{minipage}%
\end{tcolorbox}

RQ~\ref{rq-what} is a standard Research Question.

\begin{tcolorbox}[enhanced jigsaw, colframe=quarto-callout-tip-color-frame, opacityback=0, leftrule=.75mm, left=2mm, bottomrule=.15mm, toprule=.15mm, colback=white, rightrule=.15mm, arc=.35mm, breakable]
\begin{minipage}[t]{5.5mm}
\textcolor{quarto-callout-tip-color}{\faLightbulb}
\end{minipage}%
\begin{minipage}[t]{\textwidth - 5.5mm}

\quartocalloutrq{rq-what} 

\vspace{-3mm}\textbf{RQ \ref*{rq-what}: What are Research Questions?}\vspace{3mm}

This is a standard Research Question, that you might want to use in any
individual chapter. It can be cross-referenced using the custom
\texttt{rq} key.

\end{minipage}%
\end{tcolorbox}




\end{document}
